%%%%%%%%%%%%%%%%%%%%%%%%%%%%%%%%%%%%%%%%%%%%%%%%%%%%%%%%%%%%%%%%%%%%%%%%%%%%
\documentclass[12pt,a4paper]{beamer}           % pdfTeX の場合


%\usetheme{Warsaw}
%\usepackage[whole]{bxcjkjatype}% whole 指定 日本語使える

\definecolor{nico}{rgb}{1, 0.90, 0.96}
\definecolor{orange}{rgb}{0.3, 0.0, 0.8}


\usepackage[absolute, overlay]{textpos}
\usepackage{tikz}
\newcommand{\rect}[5]{
	\TPoptions{absolute=true}
	\begin{textblock*}{\textwidth}(#1pt,#2pt)
		\begin{tikzpicture}
			\draw[#5,thick] (0pt,0pt) rectangle (#3pt,#4pt);
		\end{tikzpicture}
	\end{textblock*}
	\TPoptions{absolute=false}
}
%%%%%%%%%%%%%%%%%%%%%%%%%%%%%%%%%%%%%%%%%%%%%%%%%%%%%%%%%%%%%%%%%%%%%%%%%%%

\begin{document}

% 和文タイトル
\title{Title}
\author{横山侑政}

\maketitle

%%%%%%%%%%%%%%%%%%%%%%%%%%%%%%%%%%%%%%


%------------------------------
\begin{frame}{卒論のテーマ}
	\begin{itemize}
		\item 分子の性質
		\item Deep
		\item 既存
	\end{itemize}
\end{frame}

%------------------------------
\begin{frame}{目次}
	\begin{enumerate}
			\color{blue}
		\item \textcolor{orange}{hoge}
		\item fuga
			\begin{enumerate}
					\color{blue}
				\item hoge
			\end{enumerate}
		\item hoge
		\item hoge
		\item hoge
	\end{enumerate}
\end{frame}

%------------------------------
\begin{frame}[t]{背景}
	\hskip -20pt \structure{勇気} \\
	hogehoge \\
	\structure{勇気} \\
	\hskip 100pt hogehoge \\
	\pause
	\vskip 40pt
	\structure{勇気} \\
	hogehoge \\
\end{frame}

%------------------------------
\begin{frame}{only}
	\uncover<1,3>{hogehoge} 
	\only<2-3>{fugafuga} 
	\only<3>{piyopiyo} 
\end{frame}

%------------------------------
\begin{frame}{ihoge}
	%\includegraphics[width=0.8\textwidth]{img/}
	\rect{40}{40}{80}{80}{red}
\end{frame}

%------------------------------
\begin{frame}{}
	\begin{textblock*}{\textwidth}(100pt,100pt)
		hogehoge
	\end{textblock*}
\end{frame}


%------------------------------
\begin{frame}{end}
	end
\end{frame}




%%%%%%%%%%%%%%%%%%%%%%%%%%%%%%%%%%%%%%

\end{document}

